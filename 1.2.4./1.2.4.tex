\documentclass[a4paper,11pt]{article}

\usepackage{amsmath,amsthm,amssymb}
\usepackage{graphicx}
\usepackage[warn]{mathtext}
\usepackage[T1,T2A]{fontenc}
\usepackage[utf8]{inputenc}
\usepackage[english,russian]{babel}
\usepackage{pgfplots}
\pgfplotsset{width=10cm,compat=1.9}

\title{Отчет о выполнении работы №1.2.4.}
\author{Воейко Андрей Александрович, Б01-109}
\date{Долгопрудный, 2021}

\begin{document}
\maketitle
\newpage
\section{Аннотация}
В работе измеряются периоды крутильных колебаний рамки при различных положениях закрепленного в ней тела с целью проверить теоретичесткую зависимость между периодами колебаний тела относительно нескольких осей, определить моменты инерции относительно различных осей, определить моменты инерции относительно нескольких осей для кождого тела, по ним найти главные моменты инерции.
\section{Теоретические сведения}
Тензор инерции твердого тела -- симметричная матрица, полностью определяемая шестью элементами. Из этого следует, что эту матрицу можно привести к диагональному виду. Диагональные элементы называются главными. Геометрическим образом тензора называют эллипсоид, описываемый формулой~\ref{eq1}, так называемый эллипсоид энерции.\newline
\begin{equation}    \label{eq1}
  I_{x}x^{2} + I_{y}y^{2}+ I_{z}z^{2} = 1.
\end{equation}
Координатные оси Ox, Oy, Oz совпадают с главными осями тела. Если начало координат O совпадает с центром масс, то эллипсоид нызывают центральным. Сам эллипсоид же позволяет определить момент инерции относительно любой оси. Для этого проводится радиус-вектор из центра координат до пересечения с эллипсоидом вдоль этой оси. Длина r определяет момент инерции:\newline
\begin{equation}    \label{eq2}
  I = \frac{1}{r^{2}}.
\end{equation}
В данной работе используется устройство для получения крутильных колебаний, представляющее из себя рамку, закрепленную между потолком и столом двумя нитями, в которой может зажиматься груз. Крутильные колебания рамки с телом описываются уравнением~\ref{eq3}.\newline
\begin{equation}    \label{eq3}
  (I + I_{р}) \frac{d^{2}\phi}{dt^{2}} = -f\phi,
\end{equation}
Здесь $I$ -- момент инерции тела, $I_{р}$ -- момент инерции рамки, $\phi$ -- угол поворота рамки, $t$ -- время, $f$ -- модуль кручения проволоки.\newline
Пеорид определим по формуле~\ref(eq4).\newline
\begin{equation}    \label{eq4}
  T = 2 \pi \frac{I+I_{р}}{f}
\end{equation}
Теперь предположим, что наше тело -- прямоугольный параллелепипед.\newline
% ВСТАВИТЬ КАРТИНКУ ПАРАЛЛЕЛЕПИПЕДА

\end{document}
