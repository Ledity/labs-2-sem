
\documentclass[a4paper,11pt]{article}

\usepackage{amsmath,amsthm,amssymb}
\usepackage{graphicx}
\usepackage[warn]{mathtext}
\usepackage[T1,T2A]{fontenc}
\usepackage[utf8]{inputenc}
\usepackage[english,russian]{babel}
\usepackage{pgfplots}
\pgfplotsset{width=10cm,compat=1.9}

\title{Отчет о выполнении работы №1.2.5.}
\author{Воейко Андрей Александрович, Б01-109}
\date{Долгопрудный, 2022}

\begin{document}
\maketitle
\newpage
\section{Аннотация}
В работе исследуется вынужденная прецессия гироскопа, устанавливается зависимость скорости вынужденной прецесии от величины момента сил, действующих на ось гироскопа, определяется скорость вращения ротора гироскопа по скорости прецесии оси гироскопа.
\section{Теоретические сведения}
Гироскопом называется вращающееся тело, момент импульса вращения которого по одной из осей значительно больше остальных. В таком случае при вращении вектора момента импульса $L_{\omega}$ относительно главной оси со скоростью $\Omega$ справедливо уравнение~\ref{eq1}.\newline
\begin{equation}    \label{eq1}
  \frac{d \overrightarrow{L_{\omega}}}{dt} = \overrightarrow{\Omega}\ \times\ \overrightarrow{L_{\omega}}.
\end{equation}
В силу того, что $\frac{d \overrightarrow{L_{\omega}}}{dt} = \overrightarrow{M}$, где $\overrightarrow{M}$ -- суммарный момент сил, действующих на тело, получаемм уравнение~\ref{eq2}.\newline
\begin{equation}    \label{eq2}
  \overrightarrow{M} = \overrightarrow{\Omega}\ \times\ \overrightarrow{L_{\omega}}.
\end{equation}
Такое медленное вращение оси гироскопа называется регулярной прецессией.\newline
Благодаря этой формуле можно вычислить модуль и направления вектора момента силы, необходимого для прецессии гироскопа с определенной скоростью или находить угловую скорость регулярной прецесии по моменту импульса вращения гироскопа и суммарному моменту действующих на него сил.\newline
В частности, если центр масс гироскопа массой $m_{г}$, ось которого наклонена на угол $\alpha$ от вертикали, не совпадает с точкой подмассаа, а находится от нее на расстоянии $l_{ц}$, модуль угловой скорости регулярной прецесии можно найти по формуле~\ref{eq3}.\newline
\begin{equation}    \label{eq3}
  \Omega = \frac{M}{L_{\omega}} = \frac{m_{г}gl_{ц} \sin\alpha}{I \omega_{г} \sin\alpha} = \frac{m_{г}gl_{ц}}{I \omega_{г}},
\end{equation}
где $I$ -- момент инерции гироскопа относительно оси вращения, а $\omega_{г}$ -- скорость углового вращения гироскопа.\newline
Для изучения регулярной прецессии используются уравновешенные гироскопы с дополнительными грузами. В таком случае скорость прецессии определяется по формуле~\ref{eq4}.\newline
\begin{equation}    \label{eq4}
  \Omega = \frac{mgl}{I \omega_{г}},
\end{equation}
Где $m$ -- масса груза, а $l$ -- расстояние от центра масс до груза.\newline
В данной работе исследуется как раз такой гироскоп. Момент инерции гироскопа определяется по периоду колебаний крутильного маятника с идентичным гироскопом в качестве груза. Период колебаний крутильного маятника зависит от момента инерции тела согласно формуле~\ref{eq5}.\newline
\begin{equation}    \label{eq5}
  T = 2 \pi \sqrt{\frac{I}{f}},
\end{equation}
Где $T$ -- период колебаний, а $f$ -- модуль кручения проволоки крутильного маятника.\newline
Чтобы избежать погрешности при использовании модуля кручения проволоки, используется цилиндр с измассатными массой и размерами. Таким образом, момент инерции гироскопа определяется по формуле~\ref{eq6}.\newline
\begin{equation}    \label{eq6}
  I_{г}=I_{ц}\frac{T_{г}^{2}}{T_{ц}^{2}},
\end{equation}
где $I_{г}$ -- момент инерции гироскопа, $I_{ц}$ - момент инерции цилиндра, $T_{г}$ -- период колебаний гироскопа, а $T_{ц}$ -- период колебаний цилиндра.\newline
Для измерения частоты вращения гироскопа используется осцилограф, подключенный к обмотке, в которой вращающийся и слегка намагниченный (как любой феррамагнетик) гироскоп наводит периодически изменяющееся с частотой его вращения индуктивную ЭДС.
\section{Оборудование}
Гироскоп, идентичный гироскопу ротор, крутильный маятник, секундомер, осцилограф, набор грузов, цилиндр, весы, штангенциркуль, линейка, генератор колебаний.\newline
Погрешности:
\begin{itemize}
        \item Секундомер -- инструментальная погрешность секундомера перекрывается погрешностью реакции человека -- $\Delta t = \pm 0,3\ с$ на включение и выключение.
        \item Весы -- инструментальная погрешность составляет $\pm 0,1\ г$, что значительно меньше интересующего нас порядка -- десятки грамм, поэтому для учета погрешностей будет использоваться погрешность округления -- $\Delta m = \pm 10\ г$.
        \item Генератор колебаний -- крайне высокоточный, но методика измерений при помощи осцилографа позволяет оценивать частоту колебаний гироскопа с точностью не более чем $\Delta \nu = \pm 1\ Гц$.
\end{itemize}
\section{Результаты измерений и обработка данных}
\subsection{Измерения}
\subsubsection{Длина рычага}
Длина рычага составляет $l_{ц} = 119\ мм$.
\subsubsection{Первая серия измерений} %%%%% 1 %%%%%
Проведем серию инзмерений периода регулярной прецессии для одного из грузиков и занесем результаты измерений в таблицу~\ref{table:tab1}.\newline
Масса грузика -- $m_{1} = 180\ г$.
\begin{table}[h!]
\centering
\begin{tabular}{ ||c|c|c|c|| }
  \hline
  № & Количество & Время & Период регулярной \\
   & оборотов $n$ & измерения $t$, $с$ & прецесии $T$, $с$ \\
  \hline
  1 & 2 & $159,4 \pm 0,6$ & $79,8 \pm 0,3$ \\
  2 & 2 & $161,2 \pm 0,6$ & $80,7 \pm 0,3$ \\
  3 & 2 & $160,7 \pm 0,6$ & $80,4 \pm 0,3$ \\
  4 & 2 & $159,7 \pm 0,6$ & $79,9 \pm 0,3$ \\
  5 & 2 & $160,1 \pm 0,6$ & $80,1 \pm 0,3$ \\
  \hline
\end{tabular}
\caption{Время и количество оборотов в первой серии измерений.}
\label{table:tab1}
\end{table}\newline
Средний период прецессии составил $\overline{T_{1}} = 80,2\ с$.\newline
Станадартная ошибка среднего -- $\sigma_{\overline{T_{1}}}= 0,2\ с$.\newline
Таким образом, период прецессии составил $T_{1} = 80,2 \pm 0,2\ с$.\newline
Угловая скорость прецесии составляет $\Omega_{1} = 0,0783 \pm 0,002\ \frac{рад}{с}$.
\subsubsection{Вторая серия измерений} %%%%% 2 %%%%%
Повторим измерения с другим грузиком и занесем результаты в таблицу~\ref{table:tab2}.\newline
Масса грузика -- $m_{2} = 219\ г$.
\begin{table}[h!]
\centering
\begin{tabular}{ ||c|c|c|c|| }
  \hline
  № & Количество & Время & Период регулярной \\
   & оборотов $n$ & измерения $t$, $с$ & прецесии $T$, $с$ \\
  \hline
  1 & 2 & $132,0 \pm 0,6$ & $66,0 \pm 0,3$ \\
  2 & 2 & $130,4 \pm 0,6$ & $65,2 \pm 0,3$ \\
  3 & 2 & $130,8 \pm 0,6$ & $65,4 \pm 0,3$ \\
  4 & 2 & $133,7 \pm 0,6$ & $66,9 \pm 0,3$ \\
  5 & 2 & $131,7 \pm 0,6$ & $65,7 \pm 0,3$ \\
  \hline
\end{tabular}
\caption{Время и количество оборотов во второй серии измерений.}
\label{table:tab2}
\end{table}\newline
Средний период прецессии составил $\overline{T_{2}} = 65,8\ с$.\newline
Станадартная ошибка среднего -- $\sigma_{\overline{T_{2}}}= 0,3\ с$.\newline
Таким образом, период прецесси составил $T_{2} = 65,8 \pm 0,3\ с$.\newline
Угловая скорость прецесии составляет $\Omega_{2} = 0,0955 \pm 0,004\ \frac{рад}{с}$.
\subsubsection{Третья серия измерений} %%%%% 3 %%%%%
Повторим измерения с третьим грузиком и занесем результаты в таблицу~\ref{table:tab3}.\newline
Масса грузика -- $m_{3} = 273\ г$.
\begin{table}[h!]
\centering
\begin{tabular}{ ||c|c|c|c|| }
  \hline
  № & Количество & Время & Период регулярной \\
   & оборотов $n$ & измерения $t$, $с$ & прецесии $T$, $с$ \\
  \hline
  1 & 2 & $104,5 \pm 0,6$ & $52,3 \pm 0,3$ \\
  2 & 2 & $106,0 \pm 0,6$ & $53,0 \pm 0,3$ \\
  3 & 2 & $103,7 \pm 0,6$ & $51,8 \pm 0,3$ \\
  4 & 2 & $104,7 \pm 0,6$ & $52,3 \pm 0,3$ \\
  5 & 2 & $104,0 \pm 0,6$ & $52,0 \pm 0,3$ \\
  \hline
\end{tabular}
\caption{Время и количество оборотов в третьей серии измерений.}
\label{table:tab3}
\end{table}\newline
Средний период прецессии составил $\overline{T_{3}} = 52,3\ с$.\newline
Станадартная ошибка среднего -- $\sigma_{\overline{T_{3}}}= 0,2\ с$.\newline
Таким образом, период прецессии составил $T_{3} = 52,3 \pm 0,2\ с$.\newline
Угловая скорость прецесии составляет $\Omega_{3} = 0,120 \pm 0,005\ \frac{рад}{с}$.
\subsubsection{Четвертая серия измерений} %%%%% 4 %%%%%
Повторим измерения с четвертым грузиком и занесем результаты в таблицу~\ref{table:tab4}.\newline
Масса грузика -- $m_{4} = 116\ г$.
\begin{table}[h!]
\centering
\begin{tabular}{ ||c|c|c|| }
  \hline
  № & Количество & Период регулярной \\
   & оборотов $n$ & прецесии $T$, $с$ \\
  \hline
  1 & 1 & $120,4 \pm 0,6$ \\
  2 & 1 & $122,1 \pm 0,6$ \\
  3 & 1 & $122,0 \pm 0,6$ \\
  4 & 1 & $120,0 \pm 0,6$ \\
  5 & 1 & $121,8 \pm 0,6$ \\
  \hline
\end{tabular}
\caption{Время и количество оборотов в четвертой серии измерений.}
\label{table:tab4}
\end{table}\newline
Средний период прецессии составил $\overline{T_{4}} = 121,3\ с$.\newline
Станадартная ошибка среднего -- $\sigma_{\overline{T_{4}}}= 0,4\ с$.\newline
Таким образом, период прецесии составил $T_{4} = 121,3 \pm 0,4\ с$.\newline
Угловая скорость прецесии составляет $\Omega_{4} = 0,0518 \pm 0,003\ \frac{рад}{с}$.
\subsubsection{Пятая серия измерений} %%%%% 5 %%%%%
Повторим измерения с четвертым грузиком и занесем результаты в таблицу~\ref{table:tab4}.\newline
Масса грузика -- $m_{5} = 142\ г$.
\begin{table}[h!]
\centering
\begin{tabular}{ ||c|c|c|c|| }
  \hline
  № & Количество & Период регулярной \\
   & оборотов $n$  & прецесии $T$, $с$ \\
  \hline
  1 & 1 & $96,9 \pm 0,6$ \\
  2 & 1 & $95,8 \pm 0,6$ \\
  3 & 1 & $96,6 \pm 0,6$ \\
  4 & 1 & $96,2 \pm 0,6$ \\
  5 & 1 & $96,4 \pm 0,6$ \\
  \hline
\end{tabular}
\caption{Время и количество оборотов в пятой серии измерений.}
\label{table:tab5}
\end{table}\newline
Средний период прецессии составил $\overline{T_{5}} = 96,2\ с$.\newline
Станадартная ошибка среднего -- $\sigma_{\overline{T_{5}}}= 0,2\ с$.\newline
Таким образом, период прецессии составил $T_{5} = 96,2 \pm 0,2\ с$.\newline
Угловая скорость прецесии составляет $\Omega_{5} = 0,0653 \pm 0,001\ \frac{рад}{с}$.
\subsubsection{Измерение момента инерции} %%%%% Момент инерции %%%%%
Масса цилиндра -- $m_{ц} = 1,62 \pm 0,01\ кг$.\newline
Диаметр цилиндра -- $d_{ц} = 79,3 \pm 0,1\ мм$.\newline
Таким образом, момент инерции цилиндра: $$I_{ц} = \frac{m_{ц}(d/2)^{2}}{2} = \frac{m_{ц}d^{2}}{8} = 1,27 \cdot 10^{-3} \pm 0,01 \cdot 10^{-3}\ кг \cdot м^{2}.$$
Проведем измерение периода колебаний этого цилиндра на крутильных весах. Результат запишем в таблицу~\ref{table:tab6}.
\begin{table}[h!]
\centering
\begin{tabular}{ ||c|c|c|c|| }
  \hline
  № & Время & Количество & Период колебаний \\
   & $t$, $с$ & колебаний $n$ & цилиндра $T_{ц}$, $с$ \\
  \hline
  1 & $88,4 \pm 0,6$ & 22 & $4,02 \pm 0,03$ \\
  2 & $87,9 \pm 0,6$ & 22 & $3,99 \pm 0,03$ \\
  3 & $89,0 \pm 0,6$ & 22 & $4,05 \pm 0,03$ \\
  3 & $88,5 \pm 0,6$ & 22 & $4,02 \pm 0,03$ \\
  5 & $88,8 \pm 0,6$ & 22 & $4,04 \pm 0,03$ \\
  \hline
\end{tabular}
\caption{Время и количество колебаний цилиндра на крутильном маятнике.}
\label{table:tab6}
\end{table}\newline
Средний период колебаний цилиндра составил $\overline{T_{ц}} = 4,02\ с$.\newline
Станадартная ошибка среднего -- $\sigma_{\overline{T_{ц}}}= 0,01\ с$.\newline
Таким образом, период колебаний цилиндра составил $T_{ц} = 70,3 \pm 0,2\ с$.\newline
Поведем те же измерения для ротора гироскопа, записав результаты измерений в таблицу~\ref{table:tab7}.
\begin{table}[h!]
\centering
\begin{tabular}{ ||c|c|c|c|| }
  \hline
  № & Время & Количество & Период колебаний \\
   & $t$, $с$ & колебаний $n$ & цилиндра $T_{ц}$, $с$ \\
  \hline
  1 & $80,7 \pm 0,6$ & 25 & $3,23 \pm 0,02$ \\
  2 & $80,9 \pm 0,6$ & 25 & $3,24 \pm 0,02$ \\
  3 & $81,3 \pm 0,6$ & 25 & $3,25 \pm 0,02$ \\
  3 & $80,5 \pm 0,6$ & 25 & $3,22 \pm 0,02$ \\
  5 & $79,8 \pm 0,6$ & 25 & $3,19 \pm 0,02$ \\
  \hline
\end{tabular}
\caption{Время и количество колебаний ротор на крутильном маятнике.}
\label{table:tab7}
\end{table}\newline
Средний период колебаний ротора составил $\overline{T_{г}} = 3,23\ с$.\newline
Станадартная ошибка среднего -- $\sigma_{\overline{T_{г}}}= 0,01\ с$.\newline
Таким образом, период колебаний цилиндра составил $T_{г} = 3,23 \pm 0,01\ с$.\newline
Момент инерции составляет, согласно формуле~\ref{eq6}, составляет
$$I_{г} = 1,27 \cdot 10^{-3}\ \frac{3,23^{2}}{4,02^{2}} = 1,05 \cdot 10^{-3}\ кг \cdot м^{2}.$$
Погрешность вычислим по формуле:
$$\Delta I_{г} = \sqrt{\left(\frac{T_{г}^{2}}{T_{ц}^{2}}\right)^{2} \cdot \left(\Delta I_{ц}\right)^{2} + \left(I_{ц} \frac{2T_{г}}{T_{ц}^{2}}\right)^{2} \cdot \sigma_{\overline{T_{г}}}^{2} + \left(-I_{ц} \frac{2T_{г}^{2}}{T_{ц}^{3}}\right)^{2} \cdot \sigma_{\overline{T_{ц}}}^{2}\ }.$$
Подставляя значаения, получаем:
$$\Delta I_{г} = 0,01 \cdot 10^{-3}\ кг \cdot м^{2}.$$
Таким образом, момент инерции -- $I = 1,05 \cdot 10^{-3} \pm 0,01 \cdot 10^{-3}\ кг \cdot м^{2}.$
\subsubsection{Частота вращения ротора} %%%%% Частота вращения ротора %%%%%
Выставим на генераторе частоты частоту генератора -- $400\ Гц$. Поскольку двигатель, вращающий ротор, асинхронный, частота вращения ротора должна быть меньше. Наша задача --, снижая частоту, добиться статичного эллипсоида на экране осцилографа сразу после отключения питания двигателя.\newline
Частота вращения ротора по осцилографу -- $\nu = 392 \pm 1\ Гц$.\newline
Угловая частота:
$$\omega = 2 \pi \nu = 2463 \pm 6 Гц = 2,46 \cdot 10^{3} \pm 0,01 \cdot 10^{3}\ \frac{рад}{c}.$$
Момент импульса гироскопа составляет $L_{\omega} = I \omega = 2,58 \pm 0,03\ \frac{кг \cdot м^{2}}{с}$
\subsection{Обработка данных}
\subsubsection{Оценка момента силы трения} %%%%% Момент сил трения %%%%%
Поскольку эксперименты по измерению периода регулярной прецесии проводились с поднятием оси на $5$-$6^{\circ}$ и до принятия осью горизонтального положения, для оценки воспользуемся результатами первой серии измерений. Там ось опустилась на $5^{\circ}$ за $112 с$. Таким образом, угловая скорость прецесии в следствие действия сил трения составляет $\Omega_{тр} =\ \sim0,04\ \frac{рад}{с}$. Тогда момент сил трения, согласно формуле~\ref{eq2},
$$M_{тр} = \Omega_{тр} \times L_{\omega} =\ \sim0,10\ Н \cdot м$$
\subsubsection{Проверка формулы~\ref{eq2}} %%%%% Проверка фомулы %%%%%
Вычислим момент веса грузика в каждой серии измерений периода колебаний по формуле $M_{гр} = m_{гр}g l_{ц}$. При помощи нее, а также средней угловой скорости гиросопа при помощи формуды~\ref{eq3} вычислим рассчетную угловую скорость прецесии $\Omega_{рассч}$, и сравним ее с полученным в ходе эксперимента значением.
\begin{table}[h!]
\centering
\begin{tabular}{ ||c|c|c|c|| }
  \hline
   & Момент & Рассчетная угловая & Найденная угловая \\
  № & веса груза & скорость прецессии & скорость прецессии \\
   & $M_{гр}$, $Н \cdot м$ & $\Omega_{рассч}$, $\frac{рад}{с}$ & $\Omega$, $\frac{рад}{с}$ \\
  \hline
  1 & $0,210 \pm 0,001$ & $0,0813 \pm 0,0013$ & $0,0783 \pm 0,002$ \\
  2 & $0,256 \pm 0,001$ & $0,0992 \pm 0,0015$ & $0,0955 \pm 0,004$ \\
  3 & $0,319 \pm 0,001$ &  $0,124 \pm 0,002$  & $0,120 \pm 0,005$ \\
  3 & $0,135 \pm 0,001$ & $0,0523 \pm 0,0010$ & $0,518 \pm 0,003$ \\
  5 & $0,166 \pm 0,001$ & $0,0643 \pm 0,0011$ & $0,0653 \pm 0,001$ \\
  \hline
\end{tabular}
\caption{Время и количество колебаний ротор на крутильном маятнике.}
\label{table:tab7}
\end{table}\newline
\section{Выводы}
В ходе работы были найдены и вычеслены угловые скорости регулярной прецессии гироскопа.\newline
С одной стороны, найденные угловые скорости прецесси вследствие действия веса грузиков довольно точно соответствуют рассчитанным (расхождение составляет $3,69 \%$ для первой серии измерений, $3,73 \%$ -- для второй, $3,23 \%$ -- для третьей, $0,96 \%$ -- для четвертой и $1,56 \%$ -- для пятой), что свидетельствует о соостветствии формулы~\ref{eq2} действительности. С другой стороны, только в ходе третьих измерений окрестности погрешностей вычисления найденных величин пересекаются, что свидетельствует об ошибке в измерении данных. Скорее всего, ошибка систематически происходила на этапе измерения периода прецессии, так как во всех случаях эта ошибка разная и в разную сторону.
\end{document}
