
\documentclass[a4paper,11pt]{article}

\usepackage{amsmath,amsthm,amssymb}
\usepackage{graphicx}
\usepackage[warn]{mathtext}
\usepackage[T1,T2A]{fontenc}
\usepackage[utf8]{inputenc}
\usepackage[english,russian]{babel}
\usepackage{pgfplots}
\pgfplotsset{width=10cm,compat=1.9}

\title{Отчет о выполнении работы №1.2.5.}
\author{Воейко Андрей Александрович, Б01-109}
\date{Долгопрудный, 2022}

\begin{document}
\maketitle
\newpage
\section{Аннотация}
В работе исследуется вынужденная прецессия гироскопа, устанавливается зависимость скорости вынужденной прецесии от величины момента сил, действующих на ось гироскопа, определяется скорость вращения ротора гироскопа по скорости прецесии оси гироскопа.
\section{Теоретические сведения}
Гироскопом называется вращающееся тело, момент импульса вращения которого по одной из осей значительно больше остальных. В таком случае при вращении вектора момента импульса $L_{\omega}$ относительно главной оси со скоростью $\Omega$ справедливо уравнение~\ref{eq1}.\newline
\begin{equation}    \label{eq1}
  \frac{d \overrightarrow{L_{\omega}}}{dt} = \overrightarrow{\Omega}\ \times\ \overrightarrow{L_{\omega}}.
\end{equation}
В силу того, что $\frac{d \overrightarrow{L_{\omega}}}{dt} = \overrightarrow{M}$, где $\overrightarrow{M}$ -- суммарный момент сил, действующих на тело, получаемм уравнение~\ref{eq2}.\newline
\begin{equation}    \label{eq2}
  \overrightarrow{M} = \overrightarrow{\Omega}\ \times\ \overrightarrow{L_{\omega}}.
\end{equation}
Такое медленное вращение оси гироскопа называется регулярной прецессией.\newline
Благодаря этой формуле можно вычислить модуль и направления вектора момента силы, необходимого для прецессии гироскопа с определенной скоростью или находить угловую скорость регулярной прецесии по моменту импульса вращения гироскопа и суммарному моменту действующих на него сил.\newline
В частности, если центр масс гироскопа массой $m_{г}$, ось которого наклонена на угол $\alpha$ от вертикали, не совпадает с точкой подвеса, а находится от нее на расстоянии $l_{ц}$, модуль угловой скорости регулярной прецесии можно найти по формуле~\ref{eq3}.\newline
\begin{equation}    \label{eq3}
  \Omega = \frac{M}{L_{\omega}} = \frac{m_{г}gl_{ц} \sin\alpha}{I \omega_{0} \sin\alpha} = \frac{m_{г}gl_{ц}}{I \omega_{0}},
\end{equation}
где $I$ -- момент инерции гироскопа относительно оси вращения, а $\omega_{0}$ -- скорость углового вращения гироскопа.\newline
Для изучения регулярной прецессии используются уравновешенные гироскопы с дополнительными грузами. В таком случае скорость прецессии определяется по формуле~\ref{eq4}.\newline
\begin{equation}    \label{eq4}
  \Omega = \frac{mgl}{I \omega_{0}},
\end{equation}
Где $m$ -- масса груза, а l -- расстояние от центра масс (точки подвеса) до груза.\newline
В данной работе исследуется как раз такой гироскоп. Момент инерции гироскопа определяется по периоду колебаний крутильного маятника с идентичным гироскопом в качестве груза. Период колебаний крутильного маятника зависит от момента инерции тела согласно формуле~\ref{eq5}.\newline
\begin{equation}    \label{eq5}
  T = 2 \pi \sqrt{\frac{I}{f}},
\end{equation}
Где $T$ -- период колебаний, а $f$ -- модуль кручения проволоки крутильного маятника.\newline
Чтобы избежать погрешности при использовании модуля кручения проволоки, используется цилиндр с известными массой и размерами. Таким образом, момент инерции гироскопа определяется по формуле~\ref{eq6}.\newline
\begin{equation}    \label{eq6}
  I_{0}=I_{ц}\frac{T_{0}^{2}}{T_{ц}^{2}},
\end{equation}
где $I_{0}$ -- момент инерции гироскопа, $I_{ц}$ - момент инерции цилиндра, $T_{0}$ -- период колебаний гироскопа, а $T_{ц}$ -- период колебаний цилиндра.\newline
Для измерения частоты вращения гироскопа используется осцилограф, подключенный к обмотке, в которой вращающийся и слегка намагниченный (как любой феррамагнетик) гироскоп наводит периодически изменяющееся с частотой его вращения индуктивную ЭДС.
\section{Оборудование}
Гироскоп, идентичный гироскопу ротор, крутильные весы, секундомер, осциллограф, набор грузов, цилиндр известной массы, штангенциркуль, линейка.
\section{Результаты измерений и обработка данных}
\subsection{Измерения}
\subsection{Обработка данныз}
\section{Выводы}

\end{document}
