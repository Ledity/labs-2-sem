% Document settings
\documentclass[a4paper,11pt]{article}

% Packages
  % math formulas
\usepackage{amsmath,amsthm,amssymb}
  % graphics
\usepackage{graphicx}
\usepackage{wrapfig}
  % plots
\usepackage{pgfplots}
  % other
\usepackage[warn]{mathtext}
\usepackage{cmap}
\usepackage[T1,T2A]{fontenc}
\usepackage[utf8]{inputenc}
\usepackage[english,russian]{babel}

% Package settings
%% graphicx
\graphicspath{{Pictures/}}
\DeclareGraphicsExtensions{.pdf,.png,.jpg}
%% pgfplots
\pgfplotsset{width=10cm,compat=1.9}

% Title
\title{Отчет о выполнении работы №2.1.6\\Эффект Джоуля-Томпсона}
\author{Воейко Андрей Александрович, Б01-109}
\date{Долгопрудный, 2022}

% Document
\begin{document}
\maketitle
\newpage
%%%%% АННОТАЦИЯ %%%%%
\section{Аннотация.}
В работе определяется изменение температуры углекислого газа при протекании через малопроницаемую перегородку при разных начальных значениях давления и температуры. Также вычисляются коэффиценты $a$ и $b$ уравнения Ван-Дер-Ваальса.
%%%%% ТЕОРЕТИЧЕСКИЕ СВЕДЕНИЯ %%%%%
\section{Теоретические сведения.}
Эффект Джоуля-Томпсона -- это изменение температуры газа, медленно перетекающего из области высокого давления в область низкого давления в условиях хорошей тепловой изоляции. Возникает он из-за того, что сближаясь, молекулы (или атомы) газа начинают взаимодействовать друг с другом, а именно отталкиваться, что приводит к увеличению потенциальной энергии их взаимодействия, и, как следствие, уменьшению их кинетической энергии, то есть температуры. В разреженных газах, по своим свойствам приближающимсяк идеальным, такой эффект не наблюдается. Таким образом, эффект Джоуля-Томпсона демонстрирует отличие газа от идиального.
\end{document}
