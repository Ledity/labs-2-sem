% Document settings
\documentclass[a4paper,11pt]{article}

% Packages
  % math formulas
\usepackage{amsmath,amsthm,amssymb}
  % graphics
\usepackage{graphicx}
\usepackage{wrapfig}
  % plots
\usepackage{pgfplots}
  % other
\usepackage[warn]{mathtext}
\usepackage{cmap}
\usepackage[T1,T2A]{fontenc}
\usepackage[utf8]{inputenc}
\usepackage[english,russian]{babel}

% Package settings
%% graphicx
\graphicspath{{Pictures/}}
\DeclareGraphicsExtensions{.pdf,.png,.jpg}
%% pgfplots
\pgfplotsset{width=10cm,compat=1.9}

% Title
\title{Отчет о выполнении работы №1.2.4.}
\author{Воейко Андрей Александрович, Б01-109}
\date{Долгопрудный, 2021}

% Document
\begin{document}
\maketitle
\newpage
\section{Аннотация}
В работе измеряется давление насыщенного пара жидкости при разной температуре. На основании этих данных при помощи уравнения Клапейрона-Клаузиуса вычисляется теплота пароообразования.
\section{Теоретические сведения}
Вычислить теплоту преобразования жидкости напрямую мы не будем, ведь сделать тепловые потери пренебрежимо малыми в условиях институтской лаборатории довольно сложно. Для вычисления теплоты парообразования $L$ воспользуемся уравнением Клапейрона-Клаузиуса, представленного в формуле~\ref{eq1}.
\begin{equation}    \label{eq1}
  \frac{dP}{dT}=\frac{L}{T(V_{2}-V_{1})},
\end{equation}
где $T$ -- температура пара, $P$ -- давление насыщенного пара, $L$ -- теплота парообразования жидкости, $V_{2}$ -- объем пара, $V_{1}$ -- объем жидкости. В нашем случае $V_{1}$ составляет $18 \cdot 10^{-6}\ \frac{м^{3}}{моль}$, а $V_{2}$ -- $31 \ cdot 10^{-3}\ \frac{м^{3}}{моль}$. $V_{1}$ составляет примерно $0,05\%$ от $V_{2}$, так что величиной $V_{1}$ можно пренебречь. Таким образом формула~\ref{eq1} принимает вид:
\begin{equation}    \label{eq2}
  \frac{dP}{dT}=\frac{L}{TV},
\end{equation}
где $V$ -- объем пара. Для того, чтобы выразить это $V$, воспользуемся уравнением Ван-дер-Ваальса:
\begin{equation}    \label{eq3}
  (P + \frac{a}{V^{2}})(V-b)=RT.
\end{equation}
Поскольку в работе в качетсве рабочего тела используется вода, коэффиценты $a$ и $b$ соответственно равны $0,4\ \frac{Па \cdot м^{6}}{моль^{2}}$ и $26 \cdot 10^{-6}\ \frac{м^{3}}{моль}$. Поскольку $b$ составляет менее десятой доли процента от $V$, можно ей пренебречь. Принебрежение величиной $\frac{a}{V^{2}}$ приведет к возникновению ошибки менее $3\%$, а при давлении меньше атмосферного -- еще меньше. Таким образом, уравнение Ван-дер-Ваальса для давления менее атмостферного мало отличается от уравнения Менделеева-Клапейрона.
\begin{equation}    \label{eq4}
  V = \frac{RT}{P}.
\end{equation}
Совмещая уравнения~\ref{eq1} и \ref{eq4}, получаем:
\begin{equation}    \label{eq5}
  L = \frac{RT^{2}}{P} \frac{dP}{dT} = -R \frac{d(\ln P)}{d(1 / T)}.
\end{equation}
\end{document}
