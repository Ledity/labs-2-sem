\documentclass[a4paper,11pt]{article}

\usepackage{amsmath,amsthm,amssymb}
\usepackage{graphicx}
\usepackage[warn]{mathtext}
\usepackage[T1,T2A]{fontenc}
\usepackage[utf8]{inputenc}
\usepackage[english,russian]{babel}
\usepackage{pgfplots}
\pgfplotsset{width=10cm,compat=1.9}

\title{Отчет о выполнении работы №1.3.3\\Измерение вязкости воздуха при течении в тонких трубках}
\author{Воейко Андрей Александрович, Б01-109}
\date{Долгопрудный, 2022}
\begin{document}
\maketitle
\newpage
\section{Аннотация.}
В работе экспериментально исследуется свойства течения газов по тонким трубкам, а также выявляется область применимости закона Пуассона и с его помощью определяется коэффицент вязкости воздуха.
\section{Теоретические сведения.}
Движение жидкости или газа в трубке вызвается перепадом внешнего давления $\Delta P$ на концах. Препятствуют движению силы вязкого трения, действующие между соседними слоями жидкости, а также со стороны стенок трубы. Сила вязкого трения в жидкостях и газах описывается законом Ньютона. В частности, если жидкость течет вдоль оси x, а скорость течения $v_{x}(y)$ зависит от координты и $y$, в каждом слое возникает направленное по $x$ касательное напряжение:
\begin{equation}    \label{eq1}
  \tau_{xy} = -\eta \frac{\delta v_{x}}{\delta y},
\end{equation}
Где $\tau_{xy}$ -- касательное напряжение, $\eta$ -- коэффицент динамической вязкости среды.
Характер течения в трубе может быть ламинарным -- когда слои жидкости или газа не перемешиваются между собой, и турбулентным -- когда они перемешиваются. Определяется числом Рейнольдса:
\begin{equation}    \label{eq2}
  Re = \frac{\rho u a}{\eta},
\end{equation}
где $\rho$ -- плотность среды, $u$ -- характерная скорость потока, $a$ -- характерный размер системы (размер, на котором существенно меняется скорость течения). Из опыта известно, что переход к турбулентному течению для трубок круглого сечения наблюдается при $Re_{кр} \approx 10^{3}$.\\
В целях упрощения теоретической модели газа в услоаиях эксперимента можно считать несжимаемым, то есть принять плотность среды постоянной: $\rho = const$. Для газов такое приближение допустимо, если относительный перепад давления в трубе мал: $\Delta P \ll P$.
\section{Оборудование.}
Газовый счетчик -- диап. 5 л., цена деления -- 0,02.
\section{Результаты измерений и и обработка данных.}
\section{Вывод.}
\end{document}
