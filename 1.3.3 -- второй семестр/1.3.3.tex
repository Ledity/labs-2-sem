\documentclass[a4paper,11pt]{article}

\usepackage{amsmath,amsthm,amssymb}
\usepackage{graphicx}
\usepackage[warn]{mathtext}
\usepackage[T1,T2A]{fontenc}
\usepackage[utf8]{inputenc}
\usepackage[english,russian]{babel}
\usepackage{pgfplots}
\pgfplotsset{width=10cm,compat=1.9}

\title{Отчет о выполнении работы №1.3.3\\Измерение вязкости воздуха при течении в тонких трубках}
\author{Воейко Андрей Александрович, Б01-109}
\date{Долгопрудный, 2022}

\begin{document}
\maketitle
\newpage
\section{Аннотация.}
В работе экспериментально исследуется свойства течения газов по тонким трубкам, а также выявляется область применимости закона Пуассона и с его помощью определяется коэффицент вязкости воздуха.
\section{Теоретические сведения.}
Движение жидкости или газа в трубке вызвается перепадом внешнего давления $\Delta P$ на концах. Препятствуют движению
\section{Оборудование.}
Газовый счетчик -- диап. 5 л., цена деления -- 0,02.
\section{Результаты измерений и и обработка данных.}
\section{Вывод.}
\end{document}
