\documentclass[a4paper,11pt]{article}

\usepackage{amsmath,amsthm,amssymb}
\usepackage{graphicx}
\usepackage[T1,T2A]{fontenc}
\usepackage[utf8]{inputenc}
\usepackage[english,russian]{babel}
\title{Отчет о выполнении работы №2.1.1.}
\author{Воейко Андрей Александрович, Б01-109}
\date{Долгопрудный, 2022}

\begin{document}
\maketitle
\newpage
\section{Аннотация}
В работе измеряется повышение температуры воздуха в зависимоти от мощности подводимого тепла и расхода при стационарном течении через трубу. После исключения тепловых потерь по результатам измерений определяется теплоемкость воздуха при постоянном давлении.
\section{Теоретические сведения}
Уравнение теплоемкости тела для какого-то процесса имеет вид:
\begin{equation}    \label{eq1}
C = \frac{\delta Q}{dT},
\end{equation}
где $C$ — теплоемкость тела, $\delta Q$ — количество теплоты, полученное телом, $dT$ — изменение температуры тела.
\end{document}
