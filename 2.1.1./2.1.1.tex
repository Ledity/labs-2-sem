\documentclass[a4paper,11pt]{article}

\usepackage{amsmath,amsthm,amssymb}
\usepackage{graphicx}
\usepackage[warn]{mathtext}
\usepackage[T1,T2A]{fontenc}
\usepackage[utf8]{inputenc}
\usepackage[english,russian]{babel}
\title{Отчет о выполнении работы №2.1.1.}
\author{Воейко Андрей Александрович, Б01-109}
\date{Долгопрудный, 2022}

\begin{document}
\maketitle
\newpage
\section{Аннотация}
В работе измеряется повышение температуры воздуха в зависимоти от мощности подводимого тепла и расхода при стационарном течении через трубу. После исключения тепловых потерь по результатам измерений определяется теплоемкость воздуха при постоянном давлении.
\section{Теоретические сведения}
Уравнение теплоемкости тела для какого-то процесса имеет вид:\newline
\begin{equation}    \label{eq1}
C = \frac{\delta Q}{dT},
\end{equation}
где $C$ — теплоемкость тела, $\delta Q$ — количество теплоты, полученное телом, $dT$ — изменение температуры тела.
В нашем же случае в качестве тела выступает воздух, а нагрев недостаточен для того, чтобы привести к значительному увеличению давления. Следовательно, в опыте измеряется теплоемкость воздуха при постоянном давлении.\newline
Удельная же теплоемкость определятеся по следующей формуле:\newline
\begin{equation}    \label{eq2}
c_{p} = \frac{N - N_{пот}}{q \Delta T},
\end{equation}
где $c_{p}$ — удельная теплоемкость воздуха при постоянном давлении, $N$ и $N_{пот}$ — мощности нагрева и потерь соответственно, $q$ — массовый расход воздуха, а $\Delta T$ — изменение температуры воздуха до и после нагрева.\newline
Изменение температуры найдем по формуле:\newline
\begin{equation}    \label{eq3}
\varepsilon = \beta \Delta T\ \ \Rightarrow\ \ \Delta T = \frac{\varepsilon}{\beta},
\end{equation}
где $\varepsilon$ — Э. Д. С., образовавшаяся на концах термопары, а $\beta = 40,7\ \frac{мкВ}{^{\circ}C}$ — чувствительность термопары при рабочем диапазоне температур (20 — 30 $^{\circ}C$).\newline
Расход воздуха найдем по формуле:\newline
\begin{equation}    \label{eq4}
q = \rho \frac{\Delta V}{\Delta t},
\end{equation}
где $\Delta t$ — время, за которое некоторый объем $\Delta V$ прошел через нагреватель, а $\rho$ — плотность воздуха, которую можно получить путем сложения плотности сухого воздуха $\rho _{0} = \frac{\mu P}{RT}$, где $P$ — атмосферное давление, $T$ — температура воздуха, $\mu = 29,0 \frac{г}{моль}$ — средняя молярная масса сухого воздуха; и абсолютной влажности воды $\rho _{в} = \phi \rho _{\max}$, где $\phi$ — относительная влажность, $\rho _{\max}$ — максимальная влажность воздуха при данной температуре воздуха: $\rho = \rho _{0} + \rho _{в}$.\newline
Предполагая, что в условиях, когда $\Delta T << T$, зависимость мощности потерь $N_{пот}$ от изменения температуры $\Delta T$ можно считать линейной, получаем:\newline
\begin{equation}    \label{eq5}
N_{пот} = \alpha \Delta T,
\end{equation}
где $\alpha$ — некоторая константа.\newline
Поскольку вся электрическая мощность нагревателя расходуется на нагрев проходящего воздуха и на потери, справедливо следующее уравнение:\newline
\begin{equation}    \label{eq6}
N = c_{p}q \Delta T + N_{пот} = (c_{p}q +\alpha) \Delta T
\end{equation}
Отсюда можно получить $c_{p}$:
\begin{equation}    \label{eq7}
c_{p} = \frac{N / \Delta T - \alpha}{q}
\end{equation}
\section{Оборудование и экспериментальные погрешности}
В работе используются: теплоизолированная стеклянная трубка; электронагреватель; источник питания постоянного тока; амперметр; вольтметр; термопара, подключенная к микровольтметру; компрессор; газовый счётчик; секундомер.\newline
\textbf{Амперметр} $\Delta _{A} = $\newline
\textbf{Вольтметр} $\Delta _{U} = $\newline
\textbf{Микровольтметр} $\Delta _{M} = $\newline
\textbf{Газовый счётчик} $\Delta _{V} = $\newline
\textbf{Термометр} $\Delta _{T} = 0,2\ ^{\circ}C = 0,2\ К$
\section{Результаты измерений и и обработка данных}
\subsection{Измерение температуры, давления и влажности}
Измерим температуру воздуха и его влажность, используя термометры психрометра.\newline\newline
$T = 24,2\ ^{\circ} C \pm 0,2\ ^{\circ} C = 297,2\ К \pm 0,2\ К$\newline\newline
$\phi = 48\% \pm 2\%$\newline\newline
Измерим давление при помощи цифрового барометра.\newline\newline
$P_{a} = 734,6\ мм\ рт.\ ст. \pm 0,1\ мм\ рт.\ ст. = 9,77 \cdot 10^{4}\ Па \pm 0,01 \cdot 10^{4}\ Па$\newline\newline
Найдем плотность воздуха в комнате.\newline\newline
$\rho_{0} = \frac{\mu P_{a}}{RT} = \frac{29\ \cdot\ 9,77 \cdot 10^{4}}{8,31\ \cdot\ 297,2} = 1,15 \cdot 10^{3}\ \frac{г}{м^{3}} = 1,15 \frac{кг}{м^{3}}$\newline\newline
$\Delta \rho_{0} = \frac{\mu}{R} \cdot \frac{P_{a} \Delta T + T \Delta P_{a}}{T^{2}} = \frac{29}{8,31} \cdot \frac{9,77 \cdot 10^{4}\ \cdot\ 0,2 + 297,2\ \cdot\ 0,01 \cdot 10^{4}}{297,2^{2}} = 2\ \frac{г}{м^{3}}$\newline\newline
$\rho_{в} = \phi \rho_{\max} = 0,48\ \cdot\ 20,57 = 9,87\ \frac{г}{м^{3}} = 0,01\ \frac{кг}{м^{3}}$\newline\newline
$\Delta \rho_{в}= \rho_{\max}\Delta \phi = 0,02\ \cdot\ 20,57 = 0,4 \frac{г}{м^{3}}$\newline\newline
$\rho = \rho_{0} + \rho_{в} = 1,15 + 0,01 = 1,16\ \frac{кг}{м^{3}}$\newline\newline
$\Delta \rho = \Delta \rho_{0} + \Delta \rho_{в} = 2,4 \frac{г}{м^{3}}$\newline\newline
Погрешность вычисления плотности воздуха оказалась незначительной по сравнению с интересующим нас порядком величины, поэтому в дальнейших расчетах учитываться не будет.
\subsection{Первая серия измерений, с максимальным расходом воздуха}
\subsubsection{Измерение расхода воздуха}
Измерения расхода произведем путем измерения времени, за которое через счетчик пройдет $5\ л$ воздуха.\newline
За погрешность измерения времени будем считать среднюю скорость реакции человека — $0,3\ с$.\newline
Погрешностью измерения объема будем считать $0,1\ л$ — цену деления счетчика.\newline
Результаты занесем в таблицу 1.1.\newline
В нее же занесем расход воздуха, вычесленный по формуле (1), и с погрешностью, вычесленной по следующей формуле: $\Delta q = \rho \frac{V \Delta t + t \Delta V}{t^{2}}$.\newline
\begin{center}
\begin{tabular}{ |c|c|c| }
 \hline
 № & Время, c & Расход, $\frac{кг}{с}$ \\
 \hline
 1 & $24,7 \pm 0,3$ &  \\
 2 & $25,3 \pm 0,3$ &  \\
 3 & $25,1 \pm 0,3$ &  \\
 4 & $25,1 \pm 0,3$ &  \\
 5 & $25,1 \pm 0,3$ &  \\
 \hline
\end{tabular}
\end{center}
\end{document}
