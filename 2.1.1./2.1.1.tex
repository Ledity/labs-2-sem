\documentclass[a4paper,11pt]{article}

\usepackage{amsmath,amsthm,amssymb}
\usepackage{graphicx}
\usepackage[T1,T2A]{fontenc}
\usepackage[utf8]{inputenc}
\usepackage[english,russian]{babel}
\title{Отчет о выполнении работы №2.1.1.}
\author{Воейко Андрей Александрович, Б01-109}
\date{Долгопрудный, 2022}

\begin{document}
\maketitle
\newpage
\section{Аннотация}
В работе измеряется повышение температуры воздуха в зависимоти от мощности подводимого тепла и расхода при стационарном течении через трубу. После исключения тепловых потерь по результатам измерений определяется теплоемкость воздуха при постоянном давлении.
\section{Теоретические сведения}
Уравнение теплоемкости тела для какого-то процесса имеет вид:
\begin{equation}    \label{eq1}
C = \frac{\delta Q}{dT},
\end{equation}
где $C$ — теплоемкость тела, $\delta Q$ — количество теплоты, полученное телом, $dT$ — изменение температуры тела.
В нашем же случае в качестве тела выступает воздух, а нагрев недостаточен для того, чтобы привести к значительному увеличению давления. Следовательно, в опыте измеряется теплоемкость воздуха при постоянном давлении.
\newline
Удельная же теплоемкость определятеся по следующей формуле:
\begin{equation}    \label{eq2}
c_{p} = \frac{N - N_{lost}}{q \Delta T},
\end{equation}
где $c_{p}$ — удельная теплоемкость воздуха при постоянном давлении, $N$ и $N_{lost}$ — мощности нагрева и потерь соответственно, $q$ — массовый расход воздуха, а $\Delta T$ — изменение температуры воздуха до и после нагрева.
\newline
Расход воздуха найдем по формуле:
\begin{equation}    \label{eq3}
q = \rho \frac{\Delta V}{\Delta t},
\end{equation}
где $\Delta t$ — время, за которое некоторый объем $\Delta V$ прошел через нагреватель, а $\rho$ — плотность воздуха, которую можно получить путем сложения плотности сухого воздуха $\rho _{0} = \frac{\mu P}{RT}$, где $P$ — атмосферное давление, $T$ — температура воздуха, $\mu = 29,0 \frac{г}{моль}$ — средняя молярная масса сухого воздуха; и абсолютной влажности воды $\rho _{w} = \phi \rho _{\max}$, где $\phi$ — относительная влажность, $\rho _{\max}$ — максимальная влажность воздуха при данной температуре воздуха: $\rho = \rho _{0} + \rho _{w}$.
\newline
Предполагая, что в условиях, когда $\Delta T << T$, зависимость мощности потерь $N_{lost}$ от изменения температуры $\Delta T$ можно считать линейной, получаем:
\begin{equation}    \label{eq4}
N_{lost} = \alpha \Delta T,
\end{equation}
где $\alpha$ — некоторая константа.
\newline
Поскольку вся электрическая мощность нагревателя расходуется на нагрев проходящего воздуха и на потери, справедливо следующее уравнение:
\begin{equation}    \label{eq5}
N = c_{p}q \Delta T + N_{lost} = (c_{p}q +\alpha) \Delta T
\end{equation}
Отсюда можно получить $c_{p}$:
\begin{equation}    \label{eq6}
c_{p} = \frac{N / \Delta T - \alpha}{q}
\end{equation}
\section{Оборудование и экспериментальные погрешности}
В работе используются: теплоизолированная стеклянная трубка; электронагрева-
тель; источник питания постоянного тока; амперметр; вольтметр; термопара, подключенная к микровольтметру; компрессор; газовый счётчик; секундомер.
\newline
\textbf{Амперметр}
\end{document}
