\documentclass[a4paper,11pt]{article}

\usepackage{amsmath,amsthm,amssymb}
\usepackage{graphicx}
\usepackage[warn]{mathtext}
\usepackage[T1,T2A]{fontenc}
\usepackage[utf8]{inputenc}
\usepackage[english,russian]{babel}
\usepackage{pgfplots}
\pgfplotsset{width=10cm,compat=1.9}

\title{Отчет о выполнении работы №2.1.1.}
\author{Воейко Андрей Александрович, Б01-109}
\date{Долгопрудный, 2022}

\begin{document}
\maketitle
\newpage
\section{Аннотация}
В работе измеряется повышение температуры воздуха в зависимоти от мощности подводимого тепла и расхода при стационарном течении через трубу. После исключения тепловых потерь по результатам измерений определяется теплоемкость воздуха при постоянном давлении.
\section{Теоретические сведения}
Уравнение теплоемкости тела для какого-то процесса имеет вид:\newline
\begin{equation}    \label{eq1}
C = \frac{\delta Q}{dT},
\end{equation}
где $C$ -- теплоемкость тела, $\delta Q$ -- количество теплоты, полученное телом, $dT$ -- изменение температуры тела.
В нашем же случае в качестве тела выступает воздух, а нагрев недостаточен для того, чтобы привести к значительному увеличению давления. Следовательно, в опыте измеряется теплоемкость воздуха при постоянном давлении.\newline
Удельная же теплоемкость определятеся по следующей формуле:\newline
\begin{equation}    \label{eq2}
c_{p} = \frac{N - N_{пот}}{q \Delta T},
\end{equation}
где $c_{p}$ -- удельная теплоемкость воздуха при постоянном давлении, $N$ и $N_{пот}$ -- мощности нагрева и потерь соответственно, $q$ -- массовый расход воздуха, а $\Delta T$ -- изменение температуры воздуха до и после нагрева.\newline
Изменение температуры найдем по формуле:\newline
\begin{equation}    \label{eq3}
\varepsilon = \beta \Delta T\ \ \Rightarrow\ \ \Delta T = \frac{\varepsilon}{\beta},
\end{equation}
где $\varepsilon$ -- Э. Д. С., образовавшаяся на концах термопары, а $\beta = 40,7\ \frac{мкВ}{^{\circ}C}$ -- чувствительность термопары при рабочем диапазоне температур (20 -- 30 $^{\circ}C$).\newline
Расход воздуха найдем по формуле:\newline
\begin{equation}    \label{eq4}
q = \rho \frac{\Delta V}{\Delta t},
\end{equation}
где $\Delta t$ -- время, за которое некоторый объем $\Delta V$ прошел через нагреватель, а $\rho$ -- плотность воздуха, которую можно получить путем сложения плотности сухого воздуха $\rho _{0} = \frac{\mu P}{RT}$, где $P$ -- атмосферное давление, $T$ -- температура воздуха, $\mu = 29,0 \frac{г}{моль}$ -- средняя молярная масса сухого воздуха; и абсолютной влажности воды $\rho _{в} = \phi \rho _{\max}$, где $\phi$ -- относительная влажность, $\rho _{\max}$ -- максимальная влажность воздуха при данной температуре воздуха: $\rho = \rho _{0} + \rho _{в}$.\newline
Предполагая, что в условиях, когда $\Delta T << T$, зависимость мощности потерь $N_{пот}$ от изменения температуры $\Delta T$ можно считать линейной, получаем:\newline
\begin{equation}    \label{eq5}
N_{пот} = \alpha \Delta T,
\end{equation}
где $\alpha$ -- некоторая константа.\newline
Поскольку вся электрическая мощность нагревателя расходуется на нагрев проходящего воздуха и на потери, справедливо следующее уравнение:\newline
\begin{equation}    \label{eq6}
N = c_{p}q \Delta T + N_{пот} = (c_{p}q +\alpha) \Delta T
\end{equation}
Отсюда можно получить $c_{p}$:
\begin{equation}    \label{eq7}
c_{p} = \frac{N / \Delta T - \alpha}{q}
\end{equation}
\section{Оборудование}
В работе используются: теплоизолированная стеклянная трубка; электронагреватель; источник питания постоянного тока; амперметр; вольтметр; термопара, подключенная к микровольтметру; компрессор; газовый счётчик; секундомер.\newline
\section{Результаты измерений и и обработка данных}
\subsection{Измерение температуры, давления и влажности}
Измерим температуру воздуха и его влажность, используя термометры психрометра.\newline\newline
$T = 24,2\ ^{\circ} C \pm 0,2\ ^{\circ} C = 297,2\ К \pm 0,2\ К$\newline\newline
$\phi = 48\% \pm 2\%$\newline\newline
Измерим давление при помощи цифрового барометра.\newline\newline
$P_{a} = 734,6\ мм\ рт.\ ст. \pm 0,1\ мм\ рт.\ ст. = 9,77 \cdot 10^{4}\ Па \pm 0,01 \cdot 10^{4}\ Па$\newline\newline
Найдем плотность воздуха в комнате.\newline
\begin{itemize}
  \item $\rho_{0} = \frac{\mu P_{a}}{RT} = \frac{29\ \cdot\ 9,77 \cdot 10^{4}}{8,31\ \cdot\ 297,2} = 1,15 \cdot 10^{3}\ \frac{г}{м^{3}} = 1,15 \frac{кг}{м^{3}}$
  \item $\Delta \rho_{0} = \frac{\mu}{R} \cdot \frac{P_{a} \Delta T + T \Delta P_{a}}{T^{2}} = \frac{29}{8,31} \cdot \frac{9,77 \cdot 10^{4}\ \cdot\ 0,2 + 297,2\ \cdot\ 0,01 \cdot 10^{4}}{297,2^{2}} = 2\ \frac{г}{м^{3}}$
  \item $\rho_{в} = \phi \rho_{\max} = 0,48\ \cdot\ 20,57 = 9,87\ \frac{г}{м^{3}} = 0,01\ \frac{кг}{м^{3}}$
  \item $\Delta \rho_{в}= \rho_{\max}\Delta \phi = 0,02\ \cdot\ 20,57 = 0,4 \frac{г}{м^{3}}$
  \item $\rho = \rho_{0} + \rho_{в} = 1,15 + 0,01 = 1,16\ \frac{кг}{м^{3}}$
  \item $\Delta \rho = \Delta \rho_{0} + \Delta \rho_{в} = 2,4\ \frac{г}{м^{3}}$
\end{itemize}
Погрешность вычисления плотности воздуха оказалась незначительной по сравнению с интересующим нас порядком величины, поэтому в дальнейших расчетах учитываться не будет.
\subsection{Первая серия измерений, с максимальным расходом воздуха}
\subsubsection{Измерение расхода воздуха}
Измерения расхода произведем путем измерения времени, за которое через счетчик пройдет $5\ л$ воздуха.\newline
За погрешность измерения времени будем считать среднюю скорость реакции человека -- $0,3\ с$.\newline
Погрешностью измерения объема будем считать $0,1\ л$ -- цену деления счетчика.\newline
Результаты занесем в таблицу~\ref{table:tab1}.\newline
В нее же занесем расход воздуха, вычесленный по формуле (\ref{eq1}), и с погрешностью, вычесленной по следующей формуле: $\Delta q = \rho \frac{V \Delta t}{t^{2}}$.
% Начало таблицы 1 {{{
\begin{table}[h!]
\centering
\begin{tabular}{ ||c|c|c|| }
  \hline
  № & Время $t$, $с$ & Расход $q$, $\cdot\ 10^{-2}\ \frac{г}{с}$ \\
  \hline
  1 & $24,7 \pm 0,3$ & $23,4 \pm 0,3$ \\
  2 & $25,3 \pm 0,3$ & $22,8 \pm 0,3$ \\
  3 & $25,1 \pm 0,3$ & $23,0 \pm 0,3$ \\
  4 & $25,1 \pm 0,3$ & $23,0 \pm 0,3$ \\
  5 & $25,1 \pm 0,3$ & $23,0 \pm 0,3$ \\
  \hline
\end{tabular}
\caption{Результаты изменения расхода воздуха в установке в первой серии измерений.}
\label{table:tab1}
\end{table}
% }}}
\newline
Средний расход: $\overline{q_{1}} = 0,231\ \frac{г}{с}$.\newline
Ошибка среднего: $\sigma_{\overline{q_{1}}} = \pm 0,001\ \frac{г}{с}$.\newline
Таким образом, $q_{1} = 0,231 \pm 0,001\ \frac{г}{c}$
\subsubsection{Изменение изменения температуры}
Произведем измерение изменения температуры при различных мощностях нагревателя.\newline
Результаты занесем в таблицу~\ref{table:tab2}.
% Начало таблицы 2 {{{
\begin{table}[h!]
\centering
\begin{tabular}{ ||c|c|c|c|| }
  \hline
  № & Ток через & Напряжение & Мощность \\
   & нагреватель, $мА$ & на нагревателе $V$, $В$ & нагревателя $N$, $мВт$ \\
  \hline
  1 & $166,5 \pm 0,1$ & $5,91 \pm 0,01$ & $984 \pm 2$ \\
  2 & $143,3 \pm 0,1$ & $5,07 \pm 0,01$ & $726 \pm 2$ \\
  3 & $171,7 \pm 0,1$ & $6,08 \pm 0,01$ & $1044 \pm 2$ \\
  4 & $153,8 \pm 0,1$ & $5,45 \pm 0,01$ & $838 \pm 2$ \\
  5 & $129,1 \pm 0,1$ & $4,57 \pm 0,01$ & $590 \pm 2$ \\
  \hline\hline
  № & Сопротивление & ЭДС & Разность \\
   & нагревателя $R$, $Ом$ & термопары $\epsilon$, $мкВ$ & температур $\Delta T$, $К$ \\
  \hline
  1 & $35,5 \pm 0,1$ & $144 \pm 1$ & $3,53 \pm 0,02$ \\
  2 & $35,4 \pm 0,1$ & $107 \pm 1$ & $2,63 \pm 0,02$ \\
  3 & $35,4 \pm 0,1$ & $149 \pm 1$ & $3,66 \pm 0,02$ \\
  4 & $35,4 \pm 0,1$ & $123 \pm 1$ & $3,02 \pm 0,02$ \\
  5 & $35,4 \pm 0,1$ & $90 \pm 1$ & $2,21 \pm 0,02$ \\
  \hline
\end{tabular}
\caption{Результаты изменения мощности нагревателя и ЭДС термопары в первой серии измерений.}
\label{table:tab2}
\end{table}
% }}}
\newline
В качестве погрешностей измерения тока, ЭДС термопары и напряжения указанны не инструментальные погрешности, а погрешности сокращения величин, так как инструментальные погрешности значительно менее существенны. То же самое касается и погрешности измерения сопротивления.
\subsection{Вторая серия измерений}
Уменьшим расход воздуха и повторим измерения.
\subsubsection{Расход воздуха}
Занесем результаты в таблицу~\ref{table:tab3}.
%Начало таблицы 3 {{{
\begin{table}[h!]
\centering
\begin{tabular}{ ||c|c|c|| }
  \hline
  № & Время $t$, $с$ & Расход $q$, $\cdot\ 10^{-2}\ \frac{г}{с}$ \\
  \hline
  1 & $36,5 \pm 0,3$ & $15,82 \pm 0,02$ \\
  2 & $36,3 \pm 0,3$ & $15,89 \pm 0,02$ \\
  3 & $36,5 \pm 0,3$ & $15,83 \pm 0,02$ \\
  4 & $36,5 \pm 0,3$ & $15,81 \pm 0,02$ \\
  \hline
\end{tabular}
\caption{Результаты изменения расхода воздуха в установке во второй серии измерений.}
\label{table:tab3}
\end{table}
% }}}
\newline
Средний расход: $\overline{q_{2}} = 0,158\ \frac{г}{с}$.\newline
Ошибка среднего: $\sigma_{\overline{q_{2}}} = \pm 0,0002\ \frac{г}{с}$.\newline
Ошибка среднего оказалась меньше интереующего нас порядка, в далнейших вычислениях использоваться будет погрешность округления, равная $0,001\ \frac{г}{с}$.
Таким образом, $q_{2} = 0,158 \pm 0,001 \frac{г}{с}$.
\subsubsection{Измерение изменения температуры}
Результаты занесем в таблицу~\ref{table:tab4}.
% Начало таблицы 4 {{{
\begin{table}[h!]
\centering
\begin{tabular}{ ||c|c|c|c|| }
  \hline
  № & Ток через & Напряжение & Мощность \\
   & нагреватель $I$, $мА$ & на нагревателе $V$, $В$ & нагревателя $N$, $мВт$ \\
  \hline
  1 & $121,1 \pm 0,1$ & $4,30 \pm 0,01$ & $521 \pm 2$ \\
  2 & $161,3 \pm 0,1$ & $5,73 \pm 0,01$ & $924 \pm 2$ \\
  3 & $176,1 \pm 0,1$ & $6,24 \pm 0,01$ & $1098 \pm 2$ \\
  4 & $143,7 \pm 0,1$ & $5,09 \pm 0,01$ & $732 \pm 2$ \\
  5 & $134,6 \pm 0,1$ & $4,76 \pm 0,01$ & $642 \pm 2$ \\
  \hline\hline
  № & Сопротивление & ЭДС & Разность \\
   & нагревателя $R$, $Ом$ & термопары $\epsilon$, $мкВ$ & температур $\Delta T$, $К$ \\
  \hline
  1 & $35,5 \pm 0,1$ & $103 \pm 1$ & $2,53 \pm 0,02$ \\
  2 & $35,5 \pm 0,1$ & $185 \pm 1$ & $4,54 \pm 0,02$ \\
  3 & $35,4 \pm 0,1$ & $222 \pm 1$ & $5,45 \pm 0,02$ \\
  4 & $35,4 \pm 0,1$ & $156 \pm 1$ & $3,83 \pm 0,02$ \\
  5 & $35,4 \pm 0,1$ & $136 \pm 1$ & $3,34 \pm 0,02$ \\
  \hline
\end{tabular}
\caption{Результаты изменения мощности нагревателя и ЭДС термопары во второй серии измерений.}
\label{table:tab4}
\end{table}
% }}}
\newline
В качестве погрешностей измерения тока, ЭДС термопары и напряжения указанны не инструментальные погрешности, а погрешности сокращения величин, так как инструментальные погрешности значительно менее существенны. То же самое касается и погрешности измерения сопротивления.
\subsection{Третья серия измерений}
Увеличим расход воздуха и повторим измерения.
\subsubsection{Расход воздуха}
Занесем данные в таблицу~\ref{table:tab5}.
% Начало таблицы 5 {{{
\begin{table}[h!]
\centering
\begin{tabular}{ ||c|c|c|| }
  \hline
  № & Время $t$, $с$ & Расход $q$, $\cdot\ 10^{-2}\ \frac{г}{с}$ \\
  \hline
  1 & $27,8 \pm 0,3$ & $20,8 \pm 0,2$ \\
  2 & $27,9 \pm 0,3$ & $20,7 \pm 0,2$ \\
  3 & $27,85 \pm 0,3$ & $20,7 \pm 0,2$ \\
  4 & $28,95 \pm 0,3$ & $20,6 \pm 0,2$ \\
  \hline
\end{tabular}
\caption{Результаты изменения расхода воздуха в установке в третей серии измерений.}
\label{table:tab5}
\end{table}
% }}}
\newline
Средний расход: $\overline{q_{3}} = 0,207\ \frac{г}{с}$.\newline
Ошибка среднего: $\sigma_{\overline{q_{3}}} = \pm 0,0003\ \frac{г}{с}$.\newline
Ошибка среднего оказалась меньше интереующего нас порядка, в далнейших вычислениях использоваться будет погрешность округления, равная $0,001\frac{г}{с}$.
Таким образом, $q_{3} = 0,207 \pm 0,001\ \frac{г}{с}$.
\subsubsection{Изменение изменения температуры}
Результаты занесем в таблицу~\ref{table:tab6}.
% Начало таблицы 6 {{{
\begin{table}[h!]
\centering
\begin{tabular}{ ||c|c|c|c|| }
  \hline
  № & Ток через & Напряжение & Мощность \\
   & нагреватель $I$, $мА$ & на нагревателе $V$, $В$ & нагревателя $N$, $мВт$ \\
  \hline
  1 & $131,6 \pm 0,1$ & $4,60 \pm 0,01$ & $605 \pm 2$ \\
  2 & $153,9 \pm 0,1$ & $5,40 \pm 0,01$ & $831 \pm 2$ \\
  3 & $166,75 \pm 0,1$ & $5,89 \pm 0,01$ & $981 \pm 2$ \\
  4 & $177,5 \pm 0,1$ & $6,27 \pm 0,01$ & $1112 \pm 3$ \\
  5 & $140,0 \pm 0,1$ & $4,94 \pm 0,01$ & $691 \pm 2$ \\
  \hline\hline
  № & Сопротивление & ЭДС & Разность \\
   & нагревателя $R$, $Ом$ & термопары $\epsilon$, $мкВ$ & температур $\Delta T$, $К$ \\
  \hline
  1 & $35,0 \pm 0,1$ & $87 \pm 1$ & $2,14 \pm 0,02$ \\
  2 & $35,1 \pm 0,1$ & $120 \pm 1$ & $2,95 \pm 0,02$ \\
  3 & $35,3 \pm 0,1$ & $146 \pm 1$ & $3,59 \pm 0,02$ \\
  4 & $35,3 \pm 0,1$ & $171 \pm 1$ & $4,20 \pm 0,02$ \\
  5 & $35,3 \pm 0,1$ & $110 \pm 1$ & $2,70 \pm 0,02$ \\
  \hline
\end{tabular}
\caption{Результаты изменения мощности нагревателя и ЭДС термопары в третьей серии измерений.}
\label{table:tab6}
\end{table}\newline
% }}}
\newline
В качестве погрешностей измерения тока, ЭДС термопары и напряжения указанны не инструментальные погрешности, а погрешности сокращения величин, так как инструментальные погрешности значительно менее существенны. То же самое касается и погрешности измерения сопротивления.
\subsection{Обработка данных}
\subsubsection{Зависимости разности температур от мощности нагревателя}
Построим графики зависимоти нагрева воздуха от мощности нагревателя. Аппроксимируем зависимость прямой $y=kx$, и найдем коэффицент $k$.\newline
График для первой серии измерений изображен на рисунке 1, для второй серии -- на рисунке 2, для третей серии -- на рисунке 3.\newline\newline
$k_{1} = 0,00359 \pm 0,00003\ \frac{К}{мВт}$.\newline
%
\begin{center}
\begin{tikzpicture}
\begin{axis}[
	xlabel = {$N$},
	ylabel = {$\Delta T$},
	minor tick num = 2
]
\addplot[
    mark size=2pt,
    only marks,
]
table {
	x    y
	984  3.53
	726  2.63
	1044 3.66
	838  3.02
	590  2.21
};
\addplot[
    no marks,
]
table {
	x    y
	500  1.795
    1200 4.308
};
\end{axis}
\end{tikzpicture}\newline
Рисунок 1: График зависимоти нагрева воздуха от мощности нагревателя в первой серии измерений.\newline
\end{center}
$k_{2} = 0,00502 \pm 0,00006\ \frac{К}{мВт}$.\newline
%
\begin{center}
\begin{tikzpicture}
\begin{axis}[
	xlabel = {$N$},
	ylabel = {$\Delta T$},
	minor tick num = 2
]
\addplot[
    blue,
    mark size=2pt,
    only marks,
]
table {
	x    y
	521  2.53
	924  4.54
	1008 5.45
	732  3.83
	642  3.34
};
\addplot[
    blue,
    no marks,
]
table {
	x    y
	500  2.51
    1200 6.024
};
\end{axis}
\end{tikzpicture}\newline
Рисунок 2: График зависимоти нагрева воздуха от мощности нагревателя во второй серии измерений.\newline
\end{center}
$k_{3} = 0,00370 \pm 0,00006\ \frac{К}{мВт}$.\newline
%
\begin{center}
\begin{tikzpicture}
\begin{axis}[
	xlabel = {$N$},
	ylabel = {$\Delta T$},
	minor tick num = 2
]
\addplot[
    red,
    mark size=2pt,
    only marks,
]
table {
	x    y
	601  2.14
	831  2.95
	981  3.59
	1112 4.20
	691  2.70
};
\addplot[
    red,
    no marks,
]
table {
	x    y
	500  1.85
    1200 4.44
};
\end{axis}
\end{tikzpicture}\newline
Рисунок 3: График зависимоти нагрева воздуха от мощности нагревателя в третьей серии измерений.\newline
\end{center}
\subsubsection{Анализ зависимости угла наклона графиков от расхода воздуха}
Занесем результаты измерения расхода воды $q$ и вычисления соответствующего коэффицента $k$ в таблицу~\ref{table:tab7}.
% Начало таблицы 7 {{{
\begin{table}[h!]
\centering
\begin{tabular}{ ||c|c|c|c|| }
  \hline
  № & Расход $q$, $\cdot\ 10^{-2}\ \frac{г}{с}$ & Коэффицент $k$, $\cdot\ 10^{-3}\ \frac{К}{мВт}$ & Величина $\frac{1}{k}$, $\frac{мВт}{К}$ \\
  \hline
  1 & $23,1 \pm 0,3$ & $3,57 \pm 0,03$ & $279 \pm 2$ \\
  2 & $15,8 \pm 0,1$ & $5,02 \pm 0,06$ & $199 \pm 2$ \\
  3 & $20,7 \pm 0,1$ & $3,70 \pm 0,06$ & $271 \pm 4$ \\
  \hline
\end{tabular}
\caption{Результаты измерения расхода воды $q$ и вычисления соответствующего коэффицента $k$, а также величины $\frac{1}{k}$.}
\label{table:tab7}
\end{table}
% }}}
\newline
Посколку теоретическая зависимость величины $1/k$ от расхода воздуха -- линейная, построим график зависимости $\frac{1}{k}$ от $q$.\newline
Построим на основе этих данных график 4.\newline\newline
\begin{center}
\begin{tikzpicture}
\begin{axis}[
	xlabel = {$q\ \cdot\ 10^{-2}$},
	ylabel = {$\frac{1}{k}$},
	minor tick num = 2
]
\addplot[
    green,
    mark size=2pt,
    only marks,
]
table {
	x    y
	23.1 279
	15.8 199
	20.7 271
};
\end{axis}
\end{tikzpicture}\newline
Рисунок 4: График зависимоти коэффицента $k$ от расхода воды.\newline
\end{center}
\subsubsection{Вычисление удельной теплоемкости воздуха при постоянном давлении и коэффицента тепловых потерь}
Аппроксимация по трем значениям будет крайне неточной, но за неимением более точных способов найти коэффиценты $c_{p}$ и $\alpha$ уравнения $\frac{1}{k} = \alpha + c_{p}q$, воспользуемся им.\newline
$c_{p} = 1,15\ \cdot\ 10^{3} \pm 0,3\ \cdot\ 10^{3}\ \frac{кДж}{г\ \cdot\ К}$.\newline\newline
$\alpha = 21 \pm 7\ \frac{мВт}{К}$.\newline\newline
Таким образом, чтобы найти отношение $\frac{N_{пот}}{N}$, нужно разделить $\alpha$ на $\frac{1}{k}$. Занесем результаты измерений в таблицу~\ref{table:tab8}.
% Начало таблицы 8 {{{
\begin{table}[h!]
\centering
\begin{tabular}{ ||c|c|c|| }
  \hline
  № & Величина $\frac{1}{k}$, $\frac{мВт}{К}$ & Коэффицент тепловых потерь $\frac{N_{пот}}{N}$, $\%$ \\
  \hline
  1 & $279 \pm 2$ & $6,64 \pm 2,26$ \\
  2 & $199 \pm 2$ & $10,6 \pm 3,64$ \\
  3 & $271 \pm 4$ & $7,75 \pm 2,70$ \\
  \hline
\end{tabular}
\caption{Результаты вычисления коэффицента тепловых потерь.}
\label{table:tab8}
\end{table}
% }}}
\section{Вывод}
В ходе работы были вычислено повышение температуры воздуха в зависимоти от мощности проводимого тепла и расхода воздуха при стационарном течении через трубу. Также была подтверждена гипотеза о том, что мощность тепловых потерь зависит от мощности нагрева линейно, а также вычислена его составляющая в расходе тепловой мощности нагревателя. Была вычислена удельная теплоемкость воздуха при постоянном давлении. Табличное значение находится в пределах погрешности, но сама погрешность слишком высока -- сказывается недостаточно большое количество данных.
\end{document}
