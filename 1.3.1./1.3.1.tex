% Document settings
\documentclass[a4paper,11pt]{article}

% Packages
  % math formulas
\usepackage{amsmath,amsthm,amssymb}
  % graphics
\usepackage{graphicx}
\usepackage{wrapfig}
  % plots
\usepackage{pgfplots}
  % other
\usepackage[warn]{mathtext}
\usepackage{cmap}
\usepackage[T1,T2A]{fontenc}
\usepackage[utf8]{inputenc}
\usepackage[english,russian]{babel}

% Package settings
%% graphicx
\graphicspath{{Pictures/}}
\DeclareGraphicsExtensions{.pdf,.png,.jpg}
%% pgfplots
\pgfplotsset{width=10cm,compat=1.9}

% Title
\title{Отчет о выполнении работы №1.3.1\\Определение модуля Юнга.}
\author{Воейко Андрей Александрович, Б01-109}
\date{Долгопрудный, 2021}

% Document
\begin{document}
\maketitle
\newpage
\section{Аннотация.}
В работе экспериментально измеряется зависимость между напряжением и деформацией  растяжения проволоки. По результатам измерений вычисляется модуль Юнга этой проволоки.
\section{Теоретические сведения.}

\section{Оборудование и экспериментальная установка.}
В работе используются:
\begin{itemize}
        \item прибор Лермонтова,
\end{itemize}
\section{Результаты измерений и обработка данных.}
\subsection{Результаты измерений.}
\subsection{Обработка данных.}
\section{Выводы.}
\end{document}
